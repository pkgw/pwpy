\documentclass[preprint]{aastex}
\usepackage{amsmath, amsfonts, amssymb}

\begin{document}

\title{Calibration of the DACOTA Ku-BAND Prototype System}
\author{Garrett "Karto" Keating \\ \today}

\begin{abstract} Calibration is tricky. Calibration of a super awesome telescope, doubly so. Here we explore some of the parameters to be explored with, well, you know.
\end{abstract}

\section{DACOTA System Overview}\label{secoverview}
Let us do some informal calculations regarding the proposed DACOTA system design. From current design specifications, a single DACOTA element for the prototype will consist of a 1.8 m primary, with a single polarization Ku-band feed (15-18 GHz). The single element is expect to have an overall efficient of $\eta=0.6$, and a system temperature of 100 K.

Combining these numbers, we find that a single element has a system equivalent flux density given by the following equation.
\begin{equation} \label{eqsefd}
S_{sys}=\frac{1}{2}G_{ant}{\eta}T_{sys}
\end{equation}

\noindent A factor of one-half is introduced due to the fact that we only detect a single polarization. $G_{ant}$ is the gain of the antenna, which for a 1.8 meter primary with $\eta=0.6$ is 1809 Jy/K. This gives an SEFD of $S_{ant}=181\text{ kJy}$. Assuming a minute of integration time\footnote{An estimate of one minute for calibration is made to conservatively limit the number of calibrators considered adequate for calibration. Obviously, more time for calibration can be allocated, which will in turn increase the number of viable calibrators (although not likely by a significant amount).}, for a single baseline we expect to see a signal to noise ratio of $SNR=\frac{S_{Cal}}{0.738\text{ Jy}}$. Assuming an array of 19 elements packed into a hex configuration, integrating over a minute with a bandwidth of 1 GHz will produce an image with a theoretical RMS noise of 39.9 mJy/beam. 

Our primary science driver - the detection of CO within star-forming regions of primordial galaxies - requires that we integrate down to a level of 1 microKelvin in a channel of velocity width of 100 km/s. The prototype will have a velocity resolution of roughly 300 km/s, although assuming that the brightness temperature constraint is the same for both velocity resolutions, a single pointing will require 1767 hours of observing time, and will produce an image with a theoretical noise RMS of 0.967 mJy/beam. For comparison, the full 264 element array with a system temperature of 50 K for all elements will require 7 hours of integration per pointing.

For a 1.8 meter primary, the DACOTA primary will have a field of view that covers roughly 0.258 square degrees, with sidelobes down to -30 dB (relative to the peak of the sythesized beam) extending over approximately 4.13 square degrees. We expect to see a total of 135 mJy of flux from point sources above the noise threshold\footnote{This number represents the total flux over the sky weighted by the primary beam}, the brightest point source that we expect to see over the extended field of view is 280 mJy. This implies that DACOTA will need to schieve a minimum dynamic range of 140 in imaging during calibration, and preferably a dynamic range of 290.

These constraints put some upper limits of viable calibrator choices, particularly with the DACTOTA prototype. Ideally, we are restricted to looking at calibrators whose fluxes are above 5.6 Jy (although ideally 11.6 Jy) if we are limited to one minute of calibration with the prototype. 
\begin{table}
\begin{center}
\label{List of Calibrators}
\begin{tabular}{|c|c||r|r|r|r|} \hline
Name & IAU Name & RA (J2000) & Dec (J2000) & 2 cm flux \\
\hline
\hline
3C84 & 0319+415 & 03h19m48s & +41d30m42s & 20.7 Jy \\
N/A & 0609-157 & 06h09m41s & -15d42m41s & 9.0 Jy \\
N/A & 0927+390 & 09h27m03s & +39d02m21s & 6.60 Jy \\
3C273 & 1229+020 & 12h29m07s & +02d03m09s & 34.0 Jy \\
3C279 & 1256-057 & 12h56m11s & -05d47m22s & 21.8 Jy \\
3C345 & 1642+398 & 16h42m59s & +39d48m37s & 13.0 Jy \\
N/A & 1733-130 & 17h33m03s & -13d04m50s & 11.0 Jy \\
N/A & 1924-292 & 19h24m51s & -29d14m30s & 17.0 Jy \\
3C446 & 2225-049 & 22h25m47s & -04d57m01s & 6.60 Jy \\
3C454.3 & 2253+161 & 22h53m58s & +16d08m54s & 15.0 Jy \\
\hline
\end{tabular}
\caption{List of viable DACOTA Prototype calibrators\label{tablecal}} 
\end{center}
\end{table}
\\
Assuming that the primary contribution to the system temperature is in the preamp, then we expect 


\end{document}

